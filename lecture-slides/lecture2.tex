% Options for packages loaded elsewhere
\PassOptionsToPackage{unicode}{hyperref}
\PassOptionsToPackage{hyphens}{url}
%
\documentclass[
  10pt,
  ignorenonframetext,
]{beamer}
\usepackage{pgfpages}
\setbeamertemplate{caption}[numbered]
\setbeamertemplate{caption label separator}{: }
\setbeamercolor{caption name}{fg=normal text.fg}
\beamertemplatenavigationsymbolsempty
% Prevent slide breaks in the middle of a paragraph
\widowpenalties 1 10000
\raggedbottom
\setbeamertemplate{part page}{
  \centering
  \begin{beamercolorbox}[sep=16pt,center]{part title}
    \usebeamerfont{part title}\insertpart\par
  \end{beamercolorbox}
}
\setbeamertemplate{section page}{
  \centering
  \begin{beamercolorbox}[sep=12pt,center]{part title}
    \usebeamerfont{section title}\insertsection\par
  \end{beamercolorbox}
}
\setbeamertemplate{subsection page}{
  \centering
  \begin{beamercolorbox}[sep=8pt,center]{part title}
    \usebeamerfont{subsection title}\insertsubsection\par
  \end{beamercolorbox}
}
\AtBeginPart{
  \frame{\partpage}
}
\AtBeginSection{
  \ifbibliography
  \else
    \frame{\sectionpage}
  \fi
}
\AtBeginSubsection{
  \frame{\subsectionpage}
}

\usepackage{amsmath,amssymb}
\usepackage{iftex}
\ifPDFTeX
  \usepackage[T1]{fontenc}
  \usepackage[utf8]{inputenc}
  \usepackage{textcomp} % provide euro and other symbols
\else % if luatex or xetex
  \usepackage{unicode-math}
  \defaultfontfeatures{Scale=MatchLowercase}
  \defaultfontfeatures[\rmfamily]{Ligatures=TeX,Scale=1}
\fi
\usepackage{lmodern}
\usetheme[]{berlin}
\ifPDFTeX\else  
    % xetex/luatex font selection
\fi
% Use upquote if available, for straight quotes in verbatim environments
\IfFileExists{upquote.sty}{\usepackage{upquote}}{}
\IfFileExists{microtype.sty}{% use microtype if available
  \usepackage[]{microtype}
  \UseMicrotypeSet[protrusion]{basicmath} % disable protrusion for tt fonts
}{}
\makeatletter
\@ifundefined{KOMAClassName}{% if non-KOMA class
  \IfFileExists{parskip.sty}{%
    \usepackage{parskip}
  }{% else
    \setlength{\parindent}{0pt}
    \setlength{\parskip}{6pt plus 2pt minus 1pt}}
}{% if KOMA class
  \KOMAoptions{parskip=half}}
\makeatother
\usepackage{xcolor}
\newif\ifbibliography
\setlength{\emergencystretch}{3em} % prevent overfull lines
\setcounter{secnumdepth}{-\maxdimen} % remove section numbering

\usepackage{color}
\usepackage{fancyvrb}
\newcommand{\VerbBar}{|}
\newcommand{\VERB}{\Verb[commandchars=\\\{\}]}
\DefineVerbatimEnvironment{Highlighting}{Verbatim}{commandchars=\\\{\}}
% Add ',fontsize=\small' for more characters per line
\usepackage{framed}
\definecolor{shadecolor}{RGB}{241,243,245}
\newenvironment{Shaded}{\begin{snugshade}}{\end{snugshade}}
\newcommand{\AlertTok}[1]{\textcolor[rgb]{0.68,0.00,0.00}{#1}}
\newcommand{\AnnotationTok}[1]{\textcolor[rgb]{0.37,0.37,0.37}{#1}}
\newcommand{\AttributeTok}[1]{\textcolor[rgb]{0.40,0.45,0.13}{#1}}
\newcommand{\BaseNTok}[1]{\textcolor[rgb]{0.68,0.00,0.00}{#1}}
\newcommand{\BuiltInTok}[1]{\textcolor[rgb]{0.00,0.23,0.31}{#1}}
\newcommand{\CharTok}[1]{\textcolor[rgb]{0.13,0.47,0.30}{#1}}
\newcommand{\CommentTok}[1]{\textcolor[rgb]{0.37,0.37,0.37}{#1}}
\newcommand{\CommentVarTok}[1]{\textcolor[rgb]{0.37,0.37,0.37}{\textit{#1}}}
\newcommand{\ConstantTok}[1]{\textcolor[rgb]{0.56,0.35,0.01}{#1}}
\newcommand{\ControlFlowTok}[1]{\textcolor[rgb]{0.00,0.23,0.31}{#1}}
\newcommand{\DataTypeTok}[1]{\textcolor[rgb]{0.68,0.00,0.00}{#1}}
\newcommand{\DecValTok}[1]{\textcolor[rgb]{0.68,0.00,0.00}{#1}}
\newcommand{\DocumentationTok}[1]{\textcolor[rgb]{0.37,0.37,0.37}{\textit{#1}}}
\newcommand{\ErrorTok}[1]{\textcolor[rgb]{0.68,0.00,0.00}{#1}}
\newcommand{\ExtensionTok}[1]{\textcolor[rgb]{0.00,0.23,0.31}{#1}}
\newcommand{\FloatTok}[1]{\textcolor[rgb]{0.68,0.00,0.00}{#1}}
\newcommand{\FunctionTok}[1]{\textcolor[rgb]{0.28,0.35,0.67}{#1}}
\newcommand{\ImportTok}[1]{\textcolor[rgb]{0.00,0.46,0.62}{#1}}
\newcommand{\InformationTok}[1]{\textcolor[rgb]{0.37,0.37,0.37}{#1}}
\newcommand{\KeywordTok}[1]{\textcolor[rgb]{0.00,0.23,0.31}{#1}}
\newcommand{\NormalTok}[1]{\textcolor[rgb]{0.00,0.23,0.31}{#1}}
\newcommand{\OperatorTok}[1]{\textcolor[rgb]{0.37,0.37,0.37}{#1}}
\newcommand{\OtherTok}[1]{\textcolor[rgb]{0.00,0.23,0.31}{#1}}
\newcommand{\PreprocessorTok}[1]{\textcolor[rgb]{0.68,0.00,0.00}{#1}}
\newcommand{\RegionMarkerTok}[1]{\textcolor[rgb]{0.00,0.23,0.31}{#1}}
\newcommand{\SpecialCharTok}[1]{\textcolor[rgb]{0.37,0.37,0.37}{#1}}
\newcommand{\SpecialStringTok}[1]{\textcolor[rgb]{0.13,0.47,0.30}{#1}}
\newcommand{\StringTok}[1]{\textcolor[rgb]{0.13,0.47,0.30}{#1}}
\newcommand{\VariableTok}[1]{\textcolor[rgb]{0.07,0.07,0.07}{#1}}
\newcommand{\VerbatimStringTok}[1]{\textcolor[rgb]{0.13,0.47,0.30}{#1}}
\newcommand{\WarningTok}[1]{\textcolor[rgb]{0.37,0.37,0.37}{\textit{#1}}}

\providecommand{\tightlist}{%
  \setlength{\itemsep}{0pt}\setlength{\parskip}{0pt}}\usepackage{longtable,booktabs,array}
\usepackage{calc} % for calculating minipage widths
\usepackage{caption}
% Make caption package work with longtable
\makeatletter
\def\fnum@table{\tablename~\thetable}
\makeatother
\usepackage{graphicx}
\makeatletter
\def\maxwidth{\ifdim\Gin@nat@width>\linewidth\linewidth\else\Gin@nat@width\fi}
\def\maxheight{\ifdim\Gin@nat@height>\textheight\textheight\else\Gin@nat@height\fi}
\makeatother
% Scale images if necessary, so that they will not overflow the page
% margins by default, and it is still possible to overwrite the defaults
% using explicit options in \includegraphics[width, height, ...]{}
\setkeys{Gin}{width=\maxwidth,height=\maxheight,keepaspectratio}
% Set default figure placement to htbp
\makeatletter
\def\fps@figure{htbp}
\makeatother

\usepackage{booktabs}
\usepackage{longtable}
\usepackage{array}
\usepackage{multirow}
\usepackage{wrapfig}
\usepackage{float}
\usepackage{colortbl}
\usepackage{pdflscape}
\usepackage{tabu}
\usepackage{threeparttable}
\usepackage{threeparttablex}
\usepackage[normalem]{ulem}
\usepackage{makecell}
\usepackage{xcolor}
\setbeamercolor{background canvas}{bg=white}
\setbeamertemplate{caption}[numbered]
\usecolortheme[named=black]{structure}
\usepackage{tikz}
\usepackage{pgfplots}
\makeatletter
\@ifpackageloaded{caption}{}{\usepackage{caption}}
\AtBeginDocument{%
\ifdefined\contentsname
  \renewcommand*\contentsname{Table of contents}
\else
  \newcommand\contentsname{Table of contents}
\fi
\ifdefined\listfigurename
  \renewcommand*\listfigurename{List of Figures}
\else
  \newcommand\listfigurename{List of Figures}
\fi
\ifdefined\listtablename
  \renewcommand*\listtablename{List of Tables}
\else
  \newcommand\listtablename{List of Tables}
\fi
\ifdefined\figurename
  \renewcommand*\figurename{Figure}
\else
  \newcommand\figurename{Figure}
\fi
\ifdefined\tablename
  \renewcommand*\tablename{Table}
\else
  \newcommand\tablename{Table}
\fi
}
\@ifpackageloaded{float}{}{\usepackage{float}}
\floatstyle{ruled}
\@ifundefined{c@chapter}{\newfloat{codelisting}{h}{lop}}{\newfloat{codelisting}{h}{lop}[chapter]}
\floatname{codelisting}{Listing}
\newcommand*\listoflistings{\listof{codelisting}{List of Listings}}
\makeatother
\makeatletter
\makeatother
\makeatletter
\@ifpackageloaded{caption}{}{\usepackage{caption}}
\@ifpackageloaded{subcaption}{}{\usepackage{subcaption}}
\makeatother
\ifLuaTeX
  \usepackage{selnolig}  % disable illegal ligatures
\fi
\usepackage{bookmark}

\IfFileExists{xurl.sty}{\usepackage{xurl}}{} % add URL line breaks if available
\urlstyle{same} % disable monospaced font for URLs
\hypersetup{
  pdftitle={Introduction to Statistics - Young Researchers Fellowship Program},
  pdfauthor={Daniel Sánchez Pazmiño},
  hidelinks,
  pdfcreator={LaTeX via pandoc}}

\title{Introduction to Statistics - Young Researchers Fellowship
Program}
\subtitle{Lecture 2 - More on descriptive statistics \& statistical data
visualization}
\author{Daniel Sánchez Pazmiño}
\date{September 2024}
\institute{Laboratorio de Investigación para el Desarrollo del Ecuador}

\begin{document}
\frame{\titlepage}

\begin{frame}{Recap}
\phantomsection\label{recap}
\begin{itemize}
\tightlist
\item
  So far, we covered univariate descriptive statistics:

  \begin{itemize}
  \tightlist
  \item
    Measures of central tendency
  \item
    Measures of dispersion
  \item
    Measures of position
  \item
    Measures of distributional shape
  \end{itemize}
\item
  We must also look at descriptive statistics in other contexts:

  \begin{itemize}
  \tightlist
  \item
    Categorical data descriptive stats
  \item
    Bivariate descriptive stats (measures of association)
  \item
    Statistical data visualization: boxplots, histograms, scatter plots,
    etc.
  \end{itemize}
\end{itemize}
\end{frame}

\section{Categorical data descriptive
statistics}\label{categorical-data-descriptive-statistics}

\begin{frame}{Describing categorical data}
\phantomsection\label{describing-categorical-data}
\begin{itemize}
\item
  Our univariate descriptive statistics applied quite well to numerical
  data.
\item
  However, for categorical data, would we be able to calculate a mean?

  \begin{itemize}
  \tightlist
  \item
    No, because categories are not numbers.
  \end{itemize}
\item
  There are specific descriptive stats, some of them which mirror
  numerical data stats, which should be reviewed for categorical data.

  \begin{itemize}
  \tightlist
  \item
    The frequency of each category
  \item
    Frequency tables
  \item
    Relative frequencies
  \end{itemize}
\end{itemize}
\end{frame}

\begin{frame}{Frequency of ocurrence}
\phantomsection\label{frequency-of-ocurrence}
\begin{itemize}
\tightlist
\item
  The frequency of ocurrence of a category is the number of times it
  appears in the dataset.
\end{itemize}

\[ f = \sum_{i=1}^{n} I(x_i = c) \]

where \(f\) is the frequency of category \(c\), \(n\) is the number of
observations, and \(I\) is the indicator function. - \(I(x_i = c)\) is 1
if \(x_i = c\) and 0 otherwise.

\begin{itemize}
\tightlist
\item
  This can be called the \emph{absolute frequency} of a category.
\end{itemize}
\end{frame}

\begin{frame}[fragile]{Frequency of ocurrence}
\phantomsection\label{frequency-of-ocurrence-1}
\begin{itemize}
\tightlist
\item
  Notice that if a variable in a dataset is categorical, it may have two
  or more categories within itself.

  \begin{itemize}
  \tightlist
  \item
    \texttt{sex} may have two categories: \texttt{male} and
    \texttt{female}
  \item
    \texttt{ethnicity} may have multiple categories: \texttt{mestizo},
    \texttt{afroecuadorian}, \texttt{indigenous}, etc.
  \end{itemize}
\item
  Each category of a categorical variable would have its own frequency
  of ocurrence.
\end{itemize}
\end{frame}

\begin{frame}[fragile]{Relative frequency}
\phantomsection\label{relative-frequency}
\begin{itemize}
\tightlist
\item
  The relative frequency of a category is the proportion of times it
  appears in the dataset.
\end{itemize}

\[ rf = \frac{f}{n} \]

where \(rf\) is the relative frequency of category \(c\), \(f\) is the
frequency of category \(c\), and \(n\) is the number of observations.

\begin{itemize}
\tightlist
\item
  This is given to you in \emph{proportion} form.

  \begin{itemize}
  \tightlist
  \item
    For example, if the relative frequency of \texttt{male} is 0.6, then
    60\% of the dataset falls under the \texttt{male} category.
  \item
    Proportions are always between 0 and 1.
  \item
    Find a percentage by multiplying by 100, however, it is recommended
    to keep it in proportion form for easier calculations.
  \end{itemize}
\end{itemize}
\end{frame}

\begin{frame}[fragile]{Frequency tables}
\phantomsection\label{frequency-tables}
\begin{itemize}
\item
  A frequency table is a table that shows the frequency of each category
  in a categorical variable.
\item
  It is a way to summarize the distribution of a categorical variable.
\item
  For example, consider the SUPERCIAS dataset. We can calculate the
  frequency of each category in the \texttt{region} variable.
\end{itemize}

\begin{longtable}[]{@{}lr@{}}
\toprule\noalign{}
Var1 & Freq \\
\midrule\noalign{}
\endhead
COSTA & 105744 \\
GALÁPAGOS & 1340 \\
ORIENTE & 7257 \\
SIERRA & 95277 \\
\bottomrule\noalign{}
\end{longtable}
\end{frame}

\begin{frame}{Frequency tables}
\phantomsection\label{frequency-tables-1}
\begin{itemize}
\item
  A frequency table can be presented with both the absolute frequency
  and the relative frequency.
\item
  The relative frequency is calculated by dividing the absolute
  frequency by the total number of observations.
\item
  The relative frequency is a proportion, so it is always between 0 and
  1.
\end{itemize}
\end{frame}

\begin{frame}[fragile]{Frequencies with R}
\phantomsection\label{frequencies-with-r}
\begin{itemize}
\tightlist
\item
  We can use the \texttt{table()} function in R to calculate the
  frequency of ocurrence of each category in a categorical variable
  (i.e.~a table of frequencies).

  \begin{itemize}
  \tightlist
  \item
    Works similarly to the numerical data \texttt{table()} function.
  \end{itemize}
\item
  Alternatively, use \texttt{count} from \texttt{dplyr} to calculate the
  frequency of ocurrence of each category in a categorical variable.

  \begin{itemize}
  \tightlist
  \item
    This is a shorthand for \texttt{group\_by()} and
    \texttt{summarize()} for a variable which isn't numerical.
  \end{itemize}
\item
  We may extract a specific category frequency by subsetting the table
  or using \texttt{pull()} from \texttt{dplyr}.
\end{itemize}
\end{frame}

\begin{frame}[fragile]{Example: SUPERCIAS dataset}
\phantomsection\label{example-supercias-dataset}
\begin{itemize}
\tightlist
\item
  The code for the previous frequency table is as follows:
\end{itemize}

\begin{Shaded}
\begin{Highlighting}[]
\NormalTok{supercias}\SpecialCharTok{$}\NormalTok{region }\SpecialCharTok{\%\textgreater{}\%}
    \FunctionTok{table}\NormalTok{()}
\end{Highlighting}
\end{Shaded}

\begin{verbatim}
.
    COSTA GALÁPAGOS   ORIENTE    SIERRA 
   105744      1340      7257     95277 
\end{verbatim}
\end{frame}

\begin{frame}[fragile]{Example: SUPERCIAS dataset}
\phantomsection\label{example-supercias-dataset-1}
\begin{itemize}
\tightlist
\item
  A tidyverse workflow for the frequency table is as follows:
\end{itemize}

\begin{Shaded}
\begin{Highlighting}[]
\DocumentationTok{\#\# Relative frequencies}

\NormalTok{supercias }\SpecialCharTok{\%\textgreater{}\%}
    \FunctionTok{count}\NormalTok{(region)}
\end{Highlighting}
\end{Shaded}

\begin{verbatim}
# A tibble: 4 x 2
  region         n
  <chr>      <int>
1 COSTA     105744
2 GALÁPAGOS   1340
3 ORIENTE     7257
4 SIERRA     95277
\end{verbatim}
\end{frame}

\begin{frame}[fragile]{R implementation for relative frequencies}
\phantomsection\label{r-implementation-for-relative-frequencies}
\begin{itemize}
\tightlist
\item
  For a relative frequency table, we may add an additional column to the
  frequency table with \texttt{mutate()}.

  \begin{itemize}
  \tightlist
  \item
    This column will be the relative frequency of each category.
  \end{itemize}
\item
  A base R implementation would be passing the \texttt{table()} call to
  \texttt{prop.table()}.
\end{itemize}
\end{frame}

\begin{frame}[fragile]{Example: SUPERCIAS dataset}
\phantomsection\label{example-supercias-dataset-2}
\begin{itemize}
\tightlist
\item
  The code for the relative frequency table is as follows:
\end{itemize}

\begin{Shaded}
\begin{Highlighting}[]
\NormalTok{supercias}\SpecialCharTok{$}\NormalTok{region }\SpecialCharTok{\%\textgreater{}\%}
    \FunctionTok{table}\NormalTok{() }\SpecialCharTok{\%\textgreater{}\%}
    \FunctionTok{prop.table}\NormalTok{()}
\end{Highlighting}
\end{Shaded}

\begin{verbatim}
.
      COSTA   GALÁPAGOS     ORIENTE      SIERRA 
0.504460495 0.006392581 0.034620119 0.454526806 
\end{verbatim}
\end{frame}

\begin{frame}[fragile]{Example: SUPERCIAS dataset}
\phantomsection\label{example-supercias-dataset-3}
\begin{itemize}
\tightlist
\item
  A tidyverse workflow for the relative frequency table is as follows:
\end{itemize}

\begin{Shaded}
\begin{Highlighting}[]
\NormalTok{supercias }\SpecialCharTok{\%\textgreater{}\%}
    \FunctionTok{count}\NormalTok{(region) }\SpecialCharTok{\%\textgreater{}\%}
    \FunctionTok{mutate}\NormalTok{(}\AttributeTok{relative\_frequency =}\NormalTok{ n }\SpecialCharTok{/} \FunctionTok{sum}\NormalTok{(n))}
\end{Highlighting}
\end{Shaded}

\begin{verbatim}
# A tibble: 4 x 3
  region         n relative_frequency
  <chr>      <int>              <dbl>
1 COSTA     105744            0.504  
2 GALÁPAGOS   1340            0.00639
3 ORIENTE     7257            0.0346 
4 SIERRA     95277            0.455  
\end{verbatim}

\begin{itemize}
\tightlist
\item
  Note how the denominator, \(n\), is the sum of the frequencies,
  \texttt{sum(n)}.
\end{itemize}
\end{frame}

\begin{frame}[fragile]{Dichotomous variables}
\phantomsection\label{dichotomous-variables}
\begin{itemize}
\tightlist
\item
  A dichotomous variable is a categorical variable with only two
  categories, which in some cases can be represented as 0 and 1.

  \begin{itemize}
  \tightlist
  \item
    These are also called binary or dummy variables.
  \end{itemize}
\item
  For example, \texttt{sex} can be represented as \texttt{male} and
  \texttt{female}, which can be coded as 0 and 1, respectively.

  \begin{itemize}
  \tightlist
  \item
    It's important to read the variables dictionary in a dataset to
    understand the coding of dichotomous variables.
  \end{itemize}
\end{itemize}
\end{frame}

\begin{frame}{Dichotomous variables}
\phantomsection\label{dichotomous-variables-1}
\begin{itemize}
\tightlist
\item
  The reason why dichotomous variables are important is that they can be
  used in statistical models.

  \begin{itemize}
  \tightlist
  \item
    It is beneficial to understand the category of interest as a 1 and
    the other category as a 0.
  \item
    We will talk more about these in other lectures and the Econometrics
    module.
  \end{itemize}
\item
  For now, know that \emph{if you take the mean of a dichotomous
  variable, you are calculating the proportion of the category of
  interest} in the dataset.
\end{itemize}

\[ f_{dic} = \bar{x} = \frac{1}{n} \sum_{i=1}^{n} x_i \]

where \(\bar{x}\) is the mean of the dichotomous variable \(dic\), \(n\)
is the number of observations, and \(x_i\) is the value of the
dichotomous variable for observation \(i\).
\end{frame}

\begin{frame}[fragile]{Dealing with dichotomous variables in R}
\phantomsection\label{dealing-with-dichotomous-variables-in-r}
\begin{itemize}
\item
  If a variable is dichotomous, we may want to recode it to its original
  values for better interpretation.

  \begin{itemize}
  \tightlist
  \item
    For example, \texttt{1} and \texttt{0} can be recoded to
    \texttt{male} and \texttt{female}, respectively.
  \end{itemize}
\item
  This can be done using \texttt{case\_when()} from \texttt{dplyr} in a
  \texttt{mutate()} call.
\item
  This would also allow you to do the reverse, recoding a categorical
  variable to a dichotomous variable.
\end{itemize}
\end{frame}

\begin{frame}[fragile]{Dealing with dichotomous variables in R}
\phantomsection\label{dealing-with-dichotomous-variables-in-r-1}
\begin{itemize}
\item
  Other solutions exist for recoding dichotomous variables, such as
  \texttt{recode()} from \texttt{dplyr} or \texttt{if\_else()} from
  \texttt{dplyr}.
\item
  However, R allows for the use of factors, which are a much more
  effective way to deal with categorical variables for statistical
  models.

  \begin{itemize}
  \tightlist
  \item
    These mantain the categories and their levels (order, if applicable
    or a numerical value) at the same time.
  \end{itemize}
\item
  We can convert a dichotomous variable to a factor using
  \texttt{as.factor()}.

  \begin{itemize}
  \tightlist
  \item
    This is a base R solution.
  \end{itemize}
\item
  The \texttt{forcats} package from the tidyverse is a specialized
  package for dealing with factors.

  \begin{itemize}
  \tightlist
  \item
    It has functions for reordering levels, recoding levels, and other
    factor-related tasks.
  \end{itemize}
\end{itemize}
\end{frame}

\begin{frame}[fragile]{Example: Dichotomous variable in SUPERCIAS
dataset}
\phantomsection\label{example-dichotomous-variable-in-supercias-dataset}
\begin{itemize}
\tightlist
\item
  We can manually create dummies for dichotomous variables in the
  SUPERCIAS dataset through a \texttt{mutate()} call and
  \texttt{if\_else()}.

  \begin{itemize}
  \tightlist
  \item
    For example, we can create a dummy for \texttt{region} being
    \texttt{SIERRA}.
  \end{itemize}
\end{itemize}

\begin{Shaded}
\begin{Highlighting}[]
\NormalTok{supercias\_dummies }\OtherTok{\textless{}{-}} 
\NormalTok{    supercias }\SpecialCharTok{\%\textgreater{}\%} 
    \FunctionTok{mutate}\NormalTok{(}\AttributeTok{region\_sierra =} \FunctionTok{if\_else}\NormalTok{(region }\SpecialCharTok{==} \StringTok{"SIERRA"}\NormalTok{, }\DecValTok{1}\NormalTok{, }\DecValTok{0}\NormalTok{))}
\end{Highlighting}
\end{Shaded}

\begin{itemize}
\tightlist
\item
  The proportion of \texttt{SIERRA} in the dataset can be calculated by
  taking the mean of the dummy variable.
\end{itemize}

\begin{Shaded}
\begin{Highlighting}[]
\NormalTok{supercias\_dummies}\SpecialCharTok{$}\NormalTok{region\_sierra }\SpecialCharTok{\%\textgreater{}\%} 
    \FunctionTok{mean}\NormalTok{()}
\end{Highlighting}
\end{Shaded}

\begin{verbatim}
[1] 0.4545268
\end{verbatim}
\end{frame}

\begin{frame}[fragile]{Example: Dichotomous variable in SUPERCIAS
dataset}
\phantomsection\label{example-dichotomous-variable-in-supercias-dataset-1}
\begin{itemize}
\tightlist
\item
  We can verify the proportion above is correct with a frequency table.
\end{itemize}

\begin{Shaded}
\begin{Highlighting}[]
\FunctionTok{table}\NormalTok{(supercias\_dummies}\SpecialCharTok{$}\NormalTok{region) }\SpecialCharTok{\%\textgreater{}\%} \FunctionTok{prop.table}\NormalTok{()}
\end{Highlighting}
\end{Shaded}

\begin{verbatim}

      COSTA   GALÁPAGOS     ORIENTE      SIERRA 
0.504460495 0.006392581 0.034620119 0.454526806 
\end{verbatim}
\end{frame}

\section{Descriptive statistics for bivariate
data}\label{descriptive-statistics-for-bivariate-data}

\begin{frame}{Measures of association}
\phantomsection\label{measures-of-association}
\begin{itemize}
\tightlist
\item
  Measures of association are used to describe the relationship between
  two variables.

  \begin{itemize}
  \tightlist
  \item
    Also called bivariate descriptive statistics.
  \end{itemize}
\item
  These are:

  \begin{itemize}
  \tightlist
  \item
    Covariance
  \item
    Correlation
  \item
    Contingency tables
  \end{itemize}
\end{itemize}
\end{frame}

\begin{frame}{Covariance}
\phantomsection\label{covariance}
\begin{itemize}
\tightlist
\item
  Covariance measures association between two variables in terms of
  their deviation from the mean.
\end{itemize}

\[ cov(x, y) = \frac{1}{n} \sum_{i=1}^{n} (x_i - \bar{x})(y_i - \bar{y}) \]

where \(cov(x, y)\) is the covariance between variables \(x\) and \(y\),
\(n\) is the number of observations, \(x_i\) and \(y_i\) are the values
of the variables for observation \(i\), and \(\bar{x}\) and \(\bar{y}\)
are the means of the variables.

\begin{itemize}
\tightlist
\item
  Covariance can be positive, negative, or zero.

  \begin{itemize}
  \tightlist
  \item
    Positive covariance means that as one variable increases, the other
    variable also increases.
  \item
    Negative covariance means that as one variable increases, the other
    variable decreases.
  \item
    Zero covariance means that there is no relationship between the
    variables.
  \end{itemize}
\end{itemize}
\end{frame}

\begin{frame}{Covariance}
\phantomsection\label{covariance-1}
\begin{itemize}
\tightlist
\item
  Covariance is not standardized, so it is difficult to interpret.

  \begin{itemize}
  \tightlist
  \item
    It is affected by the scale of the variables.
  \item
    It is difficult to compare covariances across different datasets.
  \end{itemize}
\item
  One should simply interpret the sign of the covariance, not the
  magnitude.
\end{itemize}
\end{frame}

\begin{frame}{Correlation}
\phantomsection\label{correlation}
\begin{itemize}
\tightlist
\item
  Correlation is a standardized version of covariance.

  \begin{itemize}
  \tightlist
  \item
    It is a measure of the strength and direction of the \textbf{linear
    relationship} between two variables.
  \end{itemize}
\end{itemize}

\[ corr(x, y) = \frac{cov(x, y)}{s_x s_y} \]

\begin{itemize}
\tightlist
\item
  The correlation coefficient is always between -1 and 1.

  \begin{itemize}
  \tightlist
  \item
    A correlation of 1 means that the variables are perfectly positively
    correlated.
  \item
    A correlation of -1 means that the variables are perfectly
    negatively correlated.
  \item
    A correlation of 0 means that there is no linear relationship
    between the variables.
  \end{itemize}
\end{itemize}
\end{frame}

\begin{frame}{Correlation}
\phantomsection\label{correlation-1}
\begin{itemize}
\tightlist
\item
  Notice that the correlation coefficient won't capture non-linear
  relationships.

  \begin{itemize}
  \tightlist
  \item
    For example, a correlation of 0 doesn't mean that there is no
    relationship between the variables, just that there is no linear
    relationship.
  \end{itemize}
\end{itemize}
\end{frame}

\begin{frame}[fragile]{R implementation for covariance and correlation}
\phantomsection\label{r-implementation-for-covariance-and-correlation}
\begin{itemize}
\tightlist
\item
  The \texttt{cov()} function in R calculates the covariance between two
  variables.

  \begin{itemize}
  \tightlist
  \item
    It is a base R function.
  \end{itemize}
\item
  The \texttt{cor()} function in R calculates the correlation between
  two variables.

  \begin{itemize}
  \tightlist
  \item
    It is a base R function.
  \end{itemize}
\item
  Both functions take two vectors as arguments.
\end{itemize}
\end{frame}

\begin{frame}[fragile]{Example with \texttt{mtcars}}
\phantomsection\label{example-with-mtcars}
\begin{itemize}
\tightlist
\item
  We can calculate the covariance between \texttt{mpg} and \texttt{wt}
  in the \texttt{mtcars} dataset.
\end{itemize}

\begin{Shaded}
\begin{Highlighting}[]
\NormalTok{mtcars }\SpecialCharTok{\%\textgreater{}\%} 
    \FunctionTok{select}\NormalTok{(mpg, wt) }\SpecialCharTok{\%\textgreater{}\%} 
    \FunctionTok{cov}\NormalTok{()}
\end{Highlighting}
\end{Shaded}

\begin{verbatim}
          mpg        wt
mpg 36.324103 -5.116685
wt  -5.116685  0.957379
\end{verbatim}
\end{frame}

\begin{frame}[fragile]{Example with \texttt{mtcars}}
\phantomsection\label{example-with-mtcars-1}
\begin{itemize}
\tightlist
\item
  We can calculate the correlation between \texttt{mpg} and \texttt{wt}
  in the \texttt{mtcars} dataset.
\end{itemize}

\begin{Shaded}
\begin{Highlighting}[]
\NormalTok{mtcars }\SpecialCharTok{\%\textgreater{}\%} 
    \FunctionTok{select}\NormalTok{(mpg, wt) }\SpecialCharTok{\%\textgreater{}\%} 
    \FunctionTok{cor}\NormalTok{()}
\end{Highlighting}
\end{Shaded}

\begin{verbatim}
           mpg         wt
mpg  1.0000000 -0.8676594
wt  -0.8676594  1.0000000
\end{verbatim}
\end{frame}

\begin{frame}{Correlation matrices}
\phantomsection\label{correlation-matrices}
\begin{itemize}
\item
  A correlation matrix is a table that shows the correlation between
  each pair of variables in a dataset.
\item
  It is a way to summarize the relationships between variables in a
  dataset
\item
  Often useful as an exploratory tool to understand the relationships
  between variables before more complex statistical analysis.
\end{itemize}
\end{frame}

\begin{frame}[fragile]{Example with \texttt{mtcars}}
\phantomsection\label{example-with-mtcars-2}
\begin{Shaded}
\begin{Highlighting}[]
\NormalTok{mtcars }\SpecialCharTok{\%\textgreater{}\%} 
    \FunctionTok{select}\NormalTok{(mpg, disp, hp, wt) }\SpecialCharTok{\%\textgreater{}\%} 
    \FunctionTok{cor}\NormalTok{()}
\end{Highlighting}
\end{Shaded}

\begin{verbatim}
            mpg       disp         hp         wt
mpg   1.0000000 -0.8475514 -0.7761684 -0.8676594
disp -0.8475514  1.0000000  0.7909486  0.8879799
hp   -0.7761684  0.7909486  1.0000000  0.6587479
wt   -0.8676594  0.8879799  0.6587479  1.0000000
\end{verbatim}
\end{frame}

\begin{frame}{Contingency tables}
\phantomsection\label{contingency-tables}
\begin{itemize}
\item
  A contingency table is a table that shows the frequency of each
  combination of categories in two categorical variables.
\item
  These are also called cross-tabulation tables.
\end{itemize}
\end{frame}

\begin{frame}[fragile]{Example with SUPERCIAS data}
\phantomsection\label{example-with-supercias-data}
\begin{itemize}
\tightlist
\item
  We can calculate a contingency table for the \texttt{region} and
  \texttt{ultimo\_balance} variables in the SUPERCIAS dataset.
\end{itemize}

\begin{Shaded}
\begin{Highlighting}[]
\NormalTok{supercias }\SpecialCharTok{\%\textgreater{}\%} 
    \FunctionTok{count}\NormalTok{(region, ultimo\_balance)}
\end{Highlighting}
\end{Shaded}

\begin{verbatim}
# A tibble: 108 x 3
   region ultimo_balance     n
   <chr>  <chr>          <int>
 1 COSTA  1995             122
 2 COSTA  1996             121
 3 COSTA  1997             389
 4 COSTA  1998             310
 5 COSTA  1999             218
 6 COSTA  2000             266
 7 COSTA  2001             200
 8 COSTA  2002             206
 9 COSTA  2003             253
10 COSTA  2004             328
# i 98 more rows
\end{verbatim}
\end{frame}

\begin{frame}[fragile]{R implementations}
\phantomsection\label{r-implementations}
\begin{itemize}
\item
  We may also feed two columns to the \texttt{table()} function to
  calculate a contingency table.
\item
  Depending on the context, you may want to modify the \texttt{margin}
  argument.
\item
  For example, \texttt{margin\ =\ 1} would give you the relative
  frequency of each row.
\item
  \texttt{margin\ =\ 2} would give you the relative frequency of each
  column.
\item
  The default is \texttt{margin\ =\ NULL}, which gives you the relative
  frequency of the entire table.
\end{itemize}
\end{frame}

\section{Statistical data
visualization}\label{statistical-data-visualization}

\begin{frame}{Introduction}
\phantomsection\label{introduction}
\begin{itemize}
\item
  Data visualization is an important part of data analysis itself,
  however, statistical data visualizations are specificially designed to
  work well with the methods we've just learned in descriptive
  statistics.
\item
  The following are common data visualization used for descriptive
  statistics:

  \begin{itemize}
  \tightlist
  \item
    Histograms/density plots
  \item
    Boxplots
  \item
    Cumulative distribution plots
  \item
    Scatter plots
  \end{itemize}
\end{itemize}
\end{frame}

\begin{frame}{Histograms}
\phantomsection\label{histograms}
\begin{itemize}
\item
  A histogram is a graphical representation of the distribution of a
  numerical variable.
\item
  It typically consists of bars that represent the frequency of each
  interval of the variable.
\item
  The height of the bars represents the frequency of the interval.
\item
  Very important for evaluating distributional shape!
\end{itemize}
\end{frame}

\begin{frame}[fragile]{Histograms in R}
\phantomsection\label{histograms-in-r}
\begin{itemize}
\item
  The \texttt{hist()} function in R is used to create histograms.

  \begin{itemize}
  \tightlist
  \item
    This is a base R function.
  \end{itemize}
\item
  A tidy alternative is the \texttt{geom\_histogram()} function from
  \texttt{ggplot2}.
\item
  The \texttt{geom\_density()} function from \texttt{ggplot2} can be
  used to create a density plot, which is a smoothed version of a
  histogram.
\end{itemize}
\end{frame}

\begin{frame}[fragile]{Example with \texttt{mtcars}}
\phantomsection\label{example-with-mtcars-3}
\begin{itemize}
\tightlist
\item
  We can create a histogram for the \texttt{mpg} variable in the
  \texttt{mtcars} dataset.
\end{itemize}

\begin{Shaded}
\begin{Highlighting}[]
\NormalTok{mtcars }\SpecialCharTok{\%\textgreater{}\%} 
    \FunctionTok{ggplot}\NormalTok{(}\FunctionTok{aes}\NormalTok{(}\AttributeTok{x =}\NormalTok{ mpg)) }\SpecialCharTok{+}
    \FunctionTok{geom\_histogram}\NormalTok{()}
\end{Highlighting}
\end{Shaded}
\end{frame}

\begin{frame}{Boxplots}
\phantomsection\label{boxplots}
\begin{itemize}
\item
  A boxplot is a graphical representation of the distribution of a
  numerical variable.
\item
  It consists of a box that represents the interquartile range (IQR) of
  the variable.
\item
  The line in the box represents the median of the variable.
\item
  The whiskers represent the range of the variable, excluding outliers.
\end{itemize}
\end{frame}

\begin{frame}[fragile]{Boxplots in R}
\phantomsection\label{boxplots-in-r}
\begin{itemize}
\item
  The \texttt{boxplot()} function in R is used to create boxplots with
  base R.
\item
  The tidy alternative is the \texttt{geom\_boxplot()} function from
  \texttt{ggplot2}.
\end{itemize}
\end{frame}

\begin{frame}[fragile]{Example with \texttt{mtcars}}
\phantomsection\label{example-with-mtcars-4}
\begin{Shaded}
\begin{Highlighting}[]
\NormalTok{mtcars }\SpecialCharTok{\%\textgreater{}\%} 
    \FunctionTok{ggplot}\NormalTok{(}\FunctionTok{aes}\NormalTok{(}\AttributeTok{x =} \DecValTok{1}\NormalTok{, }\AttributeTok{y =}\NormalTok{ mpg)) }\SpecialCharTok{+}
    \FunctionTok{geom\_boxplot}\NormalTok{()}
\end{Highlighting}
\end{Shaded}
\end{frame}

\begin{frame}{Scatter plots}
\phantomsection\label{scatter-plots}
\begin{itemize}
\item
  A scatter plot is a graphical representation of the relationship
  between two numerical variables.
\item
  It consists of points that represent the values of the two variables.
\item
  The x-axis represents one variable, and the y-axis represents the
  other variable.
\item
  Scatter plots are useful for identifying relationships between
  variables.
\end{itemize}
\end{frame}

\begin{frame}[fragile]{Scatter plots in R}
\phantomsection\label{scatter-plots-in-r}
\begin{itemize}
\item
  The \texttt{plot()} function in R is used to create scatter plots with
  base R.
\item
  The tidy alternative is the \texttt{geom\_point()} function from
  \texttt{ggplot2}.

  \begin{itemize}
  \tightlist
  \item
    Sometimes, we include a \texttt{geom\_smooth()} function to add a
    regression line to the scatter plot (the ``trend line'').
  \end{itemize}
\end{itemize}
\end{frame}

\begin{frame}[fragile]{Example with \texttt{mtcars}}
\phantomsection\label{example-with-mtcars-5}
\begin{Shaded}
\begin{Highlighting}[]
\NormalTok{mtcars }\SpecialCharTok{\%\textgreater{}\%} 
    \FunctionTok{ggplot}\NormalTok{(}\FunctionTok{aes}\NormalTok{(}\AttributeTok{x =}\NormalTok{ wt, }\AttributeTok{y =}\NormalTok{ mpg)) }\SpecialCharTok{+}
    \FunctionTok{geom\_point}\NormalTok{() }\SpecialCharTok{+}
    \FunctionTok{geom\_smooth}\NormalTok{(}\AttributeTok{method =} \StringTok{"lm"}\NormalTok{)}
\end{Highlighting}
\end{Shaded}
\end{frame}

\begin{frame}{Cumulative frequency/distribution plots}
\phantomsection\label{cumulative-frequencydistribution-plots}
\begin{itemize}
\item
  A cumulative frequency plot is a graphical representation of the
  cumulative frequency of a numerical variable.
\item
  It consists of a line that represents the cumulative frequency of the
  variable.
\item
  The x-axis represents the variable, and the y-axis represents the
  cumulative frequency.
\item
  Cumulative frequency plots are useful for understanding the
  distribution of a variable.
\item
  Typically, we may want to group the variable into intervals to create
  a cumulative frequency plot.
\end{itemize}
\end{frame}

\begin{frame}{Cumulative frequency in a table of frequencies}
\phantomsection\label{cumulative-frequency-in-a-table-of-frequencies}
\begin{itemize}
\item
  We can calculate the cumulative frequency of a variable by summing the
  frequencies of the variable up to a certain point.
\item
  This can be done by creating a cumulative sum of the frequencies.
\item
  The cumulative frequency of a variable is the sum of the frequencies
  of the variable up to that point.
\end{itemize}
\end{frame}

\begin{frame}[fragile]{Example with \texttt{SUPERCIAS}'s
\texttt{capital\_suscrito}}
\phantomsection\label{example-with-superciass-capital_suscrito}
\begin{itemize}
\item
  We can calculate the cumulative frequency of the
  \texttt{capital\_suscrito} variable in the SUPERCIAS dataset.
\item
  We will first group the variable into intervals, then calculate the
  cumulative frequency.
\end{itemize}

\begin{Shaded}
\begin{Highlighting}[]
\NormalTok{supercias\_with\_cumulative }\OtherTok{\textless{}{-}}
\NormalTok{    supercias }\SpecialCharTok{\%\textgreater{}\%} 
    \FunctionTok{mutate}\NormalTok{(}\AttributeTok{capital\_suscrito\_interval =} \FunctionTok{cut}\NormalTok{(capital\_suscrito, }\AttributeTok{breaks =} \FunctionTok{seq}\NormalTok{(}\DecValTok{0}\NormalTok{, }\DecValTok{100000000}\NormalTok{, }\AttributeTok{by =} \DecValTok{10000000}\NormalTok{))) }\SpecialCharTok{\%\textgreater{}\%} 
    \FunctionTok{count}\NormalTok{(capital\_suscrito\_interval) }\SpecialCharTok{\%\textgreater{}\%} 
    \FunctionTok{mutate}\NormalTok{(}\AttributeTok{cumulative\_frequency =} \FunctionTok{cumsum}\NormalTok{(n))}
\end{Highlighting}
\end{Shaded}
\end{frame}

\begin{frame}[fragile]{Example with \texttt{SUPERCIAS}'s
\texttt{capital\_suscrito}}
\phantomsection\label{example-with-superciass-capital_suscrito-1}
\begin{verbatim}
# A tibble: 11 x 3
   capital_suscrito_interval      n cumulative_frequency
   <fct>                      <int>                <int>
 1 (0,1e+07]                 208714               208714
 2 (1e+07,2e+07]                229               208943
 3 (2e+07,3e+07]                 93               209036
 4 (3e+07,4e+07]                 45               209081
 5 (4e+07,5e+07]                 16               209097
 6 (5e+07,6e+07]                 14               209111
 7 (6e+07,7e+07]                 14               209125
 8 (7e+07,8e+07]                  5               209130
 9 (8e+07,9e+07]                  4               209134
10 (9e+07,1e+08]                  8               209142
11 <NA>                         476               209618
\end{verbatim}
\end{frame}

\begin{frame}{Example with \texttt{SUPERCIAS}'s
\texttt{capital\_suscrito}}
\phantomsection\label{example-with-superciass-capital_suscrito-2}
\begin{itemize}
\item
  The relative frequency of each cumulative step can be calculated by
  dividing the cumulative frequency by the total number of observations.
\item
  This is the cumulative relative frequency. It will be between 0 and 1
  for each step.
\item
  The cumulative relative frequency for the last step will always be 1.
\end{itemize}
\end{frame}



\end{document}
